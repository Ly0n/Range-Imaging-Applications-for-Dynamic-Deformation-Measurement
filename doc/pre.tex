%
%
%	Microtype macht einige Sachen etwas sch�ner. Sie dazu die Paketbeschreibung
\usepackage[protrusion=true,expansion=true,final]{microtype}
%
%
%	Blindtext
\usepackage{blindtext} %um mit leerem text zu f�llen
%Ben�tigt um Kopfzeile je nach Kapitel zu setzen
\usepackage{xifthen}
%
%
%\usepackage[numbers,round]{natbib}
%	Coole Kopfzeilen
\usepackage{fancyhdr}
\pagestyle{fancy}	
%
%	Kolumnentitel ohne Nummer davor
\renewcommand{\sectionmarkformat}{}
%
%	Alle Felder l�schen
\fancyhf{}	
%
%	Kopfzeile nicht in Grossschrift und ohne Nummerierung
\renewcommand{\sectionmark}[1]%
	{\markboth{ {} #1}{}}
\renewcommand{\subsectionmark}[1]%
	{\markright{ {} #1}{}}
%
%	Keine Linie zur Abtrennung der Kopfzeile
\renewcommand{\headrulewidth}{0mm}
%
%	Header
\fancyhead[R]{\parbox{\textwidth}{
	\begin{flushright}\small
		\textbf{\nouppercase\leftmark\phantom{\hspace*{1mm}}}{\textbar}\\%
		\vspace*{-1.1mm} %zeilenabstand etwas k�rzen
		%zweite Zeile:		
		\ifthenelse{\dimtest{\widthof{\rightmark}}{<}{3mm}}{\phantom{\textbar}}
		{\rightmark\hspace*{1mm}{\textbar}}
	\end{flushright}
}}
%
\fancyfoot[R]{\parbox{0.60\textwidth}{\begin{flushright}%
\textbf{\thepage}\hspace*{1mm}%
\end{flushright}}\hspace{0cm}}		% Seitenzahl rechts bzw. links unten
%
%
%	1,5facher Zeilenabstand
\usepackage{setspace}
\onehalfspacing
%
%
%	Deutsche Anpassungen
\usepackage[english]{babel}
\usepackage[ansinew]{inputenc} %iso 8859-1 (latin-1) als standart
\usepackage{scrhack}
%das funktioniert mit Windows gut
%
%
% Packages f�r Grafiken & Tabellen
\usepackage{graphicx} 			%Zum Laden von Grafiken
\usepackage{ctable}				%bunte Tabellen
%
%
%	Hyperref (klickbare Links im PDF)
%\usepackage[bookmarksnumbered,draft]{hyperref} 
\usepackage[bookmarksnumbered,final]{hyperref} %links ins web und innerhalb des pdf
%%f�r den Druck diese Zeile nutzen, damit die Links nicht gesetzt werden
%\usepackage[bookmarksnumbered,ocgcolorlinks,  linkcolor=black,        citecolor=black,        filecolor=black,        urlcolor=black,]{hyperref}
%
%
%	Bricht URLs um, besonders im Literaturverz. f�r den Druck auskommentieren
%	War in meiner alten Vorlage drin, scheint man mit einem aktuellen hyperref-Paket nicht mehr zu brauchen (es sei dennm man setzt �ber DVI und PS nach PDF und nicht mit PDFLaTeX)
% \usepackage{breakurl}
%
%
%	Mehrere Bilder in einer Figure
\usepackage{subfigure}
%
%
%	Farben
\usepackage{color}
\definecolor{DarkGrey}{cmyk}{0,0,0,0.75} % f�r Marginalien
\definecolor{mint}{cmyk}{1, 1, 1, 1} %FH-Mint mit 25% Schwarz, da es bei der d�nnen Schrift sonst sehr unter geht.
%
%
%	Listing-Umgebungen
\usepackage[final]{listings}		%Quellcode-Listings. final, damit die auch bei draft gesetzt werden
\usepackage{marvosym}			%paket, f�r das Enter-Symbol
%Javascript
\lstdefinelanguage{JavaScript}{
  keywords={typeof, new, true, false, catch, function, return, null, catch, switch, var, if, in, while, do, else, case, break, document, innerHTML, getElementById}
  keywordstyle=\color{blue}\bfseries,
  ndkeywords={class, export, boolean, throw, implements, import, this},
  ndkeywordstyle=\color{darkgray}\bfseries,
  identifierstyle=\color{black},
  sensitive=false,
  comment=[l]{//},
  morecomment=[s]{/*}{*/},
  commentstyle=\color{purple}\ttfamily,
  stringstyle=\color{red}\ttfamily,
  morestring=[b]',
  morestring=[b]"
}
%
%
%	F�r \floarbarrier, damit man floating-Objekte aufhalten kann (z.B. vor einer neuen section alle floats setzen)
\usepackage{placeins}
%
%
%	Dinge
\clubpenalty = 10000 % schliesst Schusterjungen aus
\widowpenalty = 10000 % schliesst Hurenkinder aus
%
%
%	Um das Logo absolut zu Posistionieren
%	In dieser Vorlage wird das FH-Logo nicht mehr benutzt, weil es auf den Umschlag kommt
%\usepackage[absolute]{textpos}
%\setlength{\TPHorizModule}{1mm}
%\setlength{\TPVertModule}{\TPHorizModule}
%\textblockorigin{0mm}{0mm}
%
%
%	Absatzeinzug verkleinern (bzw. eliminieren)
\setlength{\parindent}{0cm} 
%
%
%	Abk�rzungsverzeichnis
\usepackage[smaller]{acronym} 
%	Siehe dazu: http://www.matthias-schlecker.de/acronym-abkuerzungsverzeichnis-mit-latex-automatisieren
% 
%
%	Manuelle Worttrennung
\hyphenation{Ad-mi-nis-tra-tor Get-Re-quest Get-Re-sponse  Get-Next-Request Get-Bulk-Request Pro-gram-mie-rung fast-ether-net hexa-de-zi-mal-en}
%	Funktioniert trotzdem nicht mit Umlauten. Da dann im Text so trennen: �ber\-nahme\-er\-kl�rung
%
%
%	�berschriften im FH-Aachen-Mint
\addtokomafont{section}{\normalfont\Huge\color{mint}\hspace*{-0cm}} 
%
%
%	Abstand vor Kapitel�berschriften
\let\stdsection\section 
\renewcommand\section{\newpage\vspace*{1cm}\stdsection}
%
%
%	Ein wenig an der Geometrie herumzupfen um in die Vorlage zu passen
\usepackage{geometry}
%	Die auskommentierte Zeile nutzen um die Rahmen der Bereiche anzuzeigen
%\geometry{a4paper,left=2.7cm,right=6.7cm,top=2.0cm,bottom=4.4cm,marginparsep=7mm,marginparwidth=3.4cm,driver=pdftex,showframe}
\geometry{a4paper,left=2.7cm,right=2.7cm,top=2.5cm,bottom=4.4cm,marginparsep=7mm,marginparwidth=3.4cm,driver=pdftex}
%
%
%	Marginalien in grau und in kleiner
%\renewcommand\marginline[1]{%
%  \marginpar[{\begin{flushleft}\small\raggedleft\textcolor{DarkGrey}{#1}\end{flushleft}}]%
%            {{\small\raggedright\textcolor{DarkGrey}{#1}}}%
%}
%
%	Ich wei� nicht mehr, wozu ich das mal gebraucht habe. Eigentlich ist das die Aufgabe von Geometry. Daher erstmal auskommentieren
%\setlength{\headheight}{48pt}
%

\usepackage{ccicons}


\usepackage{booktabs}
%
\usepackage[T1]{fontenc}
%
%
%	Iwona-Schriften
\usepackage[light,math]{iwona}
\usepackage{amsmath}
%
\usepackage{longtable}
\usepackage{caption}

% Literaturverzeichnis-Header umdefinieren, damit oben rechts nicht zwei Mal "Literatur" steht
\makeatletter
\renewcommand*\bib@heading{%
  \section*{\bibname}%
  \@mkboth{\bibname}{}}%
\makeatother