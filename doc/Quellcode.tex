\section{Source code extracts}\label{app:sourcecode}
\begin{lstlisting}[caption={Creation of variance,amplitude and frequency images},label={list:FFT}]
% FFT and variance
for row = 1:424
	for column = 1:512
		for g= 1:L
			DynPixel(g)= FramearrayDyn(uint16(row),uint16(column),g); 
		end
		DynPixel=DynPixel-mean(DynPixel);
		
		%Variance
		V = var(DynPixel) 
		
		% FFT
		Y = fft(DynPixel,NFFT)/(L);
		Y_AmpsDyn = abs(Y(1:NFFT/2+1));
		
		% Find peaks in the spectrum
		[peakamplitude, peakindex]= max(4*Y_AmpsDyn);
		freq=f(peakindex);
	
		% Color Pixel in image
		FreqMap(row,column) = freq;
		AmpMap(row,column) = peakamplitude;
		VarMap(row,column) = V;
		
   	end
end
\end{lstlisting}

\newpage 
\begin{lstlisting}[caption={Evaluation of LDV data from scope},label={list:LDV_FFT}]
% Scope and Laser Doppler Vibrometer Data
sensitivity = 125 %%(mm/s)/V
samples = 1001; % zero based
TimePerDIV = 0.2%s
sampingrate_scope = (samples-1)/(TimePerDIV*10) 

 csvFilesS=dir(strcat(ImageFolder,'\*.csv'));
 if ~isempty(csvFilesS)
 second=csvread(strcat(ImageFolder,'\',csvFilesS.name),2,0,[2,0,samples,0]);
 Volt=csvread(strcat(ImageFolder,'\',csvFilesS.name),2,1,[2,1,samples,1]);
 %Remove Offset
 speed =(Volt-mean(Volt))*sensitivity;

 %Integrate the Speed to get the displacement
 displacement=cumtrapz(speed)/sampingrate_scope;
 %Remove Offset
 displacement = displacement-mean(displacement);
 %Add time offset
 second=second+max(second);
 figure; 
 subplot(2,2,1:2);
 plot(second,displacement);
 title('Time Domain of Scope Data');
 xlabel('Time [s]');
 ylabel('Displacement [mm]');
 xlim([min(second) max(second)]);
 
 subplot(2,2,3:4);
 Lscope = length(displacement);
 NFFT_scope = 2^nextpow2(Lscope); % Next power of 2 from length of y 
 Y = fft(displacement,NFFT_scope)/(Lscope);
 f_scope = sampingrate_scope/2*linspace(0,1,NFFT_scope/2+1);
 Y_Laser = abs(Y(1:NFFT_scope/2+1));
 plot(f_scope,2*Y_Laser);
 title('Frequency Domain of Scope Data');
 xlabel('Frequency [Hz]');
 ylabel('Amplitude [mm]');
 xlim([0 15]);
 end
 \end{lstlisting}
